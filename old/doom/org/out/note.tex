% Created 2023-04-25 Tue 19:47
% Intended LaTeX compiler: pdflatex
\documentclass[11pt]{article}
\usepackage[utf8]{inputenc}
\usepackage[T1]{fontenc}
\usepackage{graphicx}
\usepackage{longtable}
\usepackage{wrapfig}
\usepackage{rotating}
\usepackage[normalem]{ulem}
\usepackage{amsmath}
\usepackage{amssymb}
\usepackage{capt-of}
\usepackage{hyperref}
\author{oneYang}
\date{\textit{<2023-04-24 Mon>}}
\title{org-note}
\hypersetup{
 pdfauthor={oneYang},
 pdftitle={org-note},
 pdfkeywords={},
 pdfsubject={},
 pdfcreator={Emacs 27.1 (Org mode 9.6.1)}, 
 pdflang={English}}
\begin{document}
\maketitle
\tableofcontents

\begin{quote}
Org Mode compact guide  \\[0pt]
Org Mode Version: 9.6
\end{quote}

\section{使用 doom emacs 中的 org mode}
\label{sec:org650ec81}
\subsection{文档结构和基本语法}
\label{sec:org38ad953}
\subsubsection{标题}
\label{sec:org8621d12}
使用 * 区分标题层级
\subsubsection{内容可见性}
\label{sec:orgf4591a2}
\begin{enumerate}
\item 子树循环
\label{sec:org5eea55c}
使用 TAB 控制子树的状态
\begin{verbatim}
,-> FOLDED -> CHILDREN -> SUBTREE --.
'-----------------------------------'
\end{verbatim}
\item 全局循环
\label{sec:org843ad56}
使用 S-TAB 控制整个 Buffer 的状态
\begin{verbatim}
,-> OVERVIEW -> CONTENTS -> SHOW ALL --.
'--------------------------------------'
\end{verbatim}
\item 控制默认的显示状态
\label{sec:orgd27d2aa}
\begin{verbatim}
#+startup: overview
# overview, content, showall, showeveryting, show<n>levels
\end{verbatim}
\end{enumerate}

\subsubsection{移动}
\label{sec:org4aac87f}
\begin{center}
\begin{tabular}{lll}
Doom-Keybinding & Keybinding & Action\\[0pt]
\hline
 & C-c C-n & 下一个标题\\[0pt]
 & C-c C-p & 上一个标题\\[0pt]
 & C-c C-f & 下一个同级标题\\[0pt]
 & C-c C-b & 上一个同级标题\\[0pt]
 & C-c C-u & 回到更高级别的标题\\[0pt]
\end{tabular}
\end{center}
\subsubsection{结构编辑}
\label{sec:orgbed7193}
\begin{center}
\begin{tabular}{lll}
Doom-Keybinding & Keybinding & Action\\[0pt]
\hline
 & M-RET & 插入同级标题或列表项\\[0pt]
 & M-S-RET & 插入同级 TODO\\[0pt]
 & M-Left / M-Right & 标题升级 / 降级\\[0pt]
 & M-Up / M-Down & 向上 / 向下移动子树\\[0pt]
 & C-c C-w & 将条目或区域重新归档到不同的位置(标题下的内容会移动到选择的位置并降级)\\[0pt]
 & C-x n s / C-x n w & 缩小子树和展开子树\\[0pt]
\end{tabular}
\end{center}
\subsubsection{稀疏树 Sparse Trees}
\label{sec:orgd993dff}
\begin{quote}
在大纲树中选中信息构建稀疏树,从而尽可能折叠整个文档,但选中的信息连通其上方的标题结构一起可见。
\end{quote}

\begin{itemize}
\item C-c /
提示输入一个额外的键来选择稀疏树的创建命令
\end{itemize}
\subsubsection{普通列表 Plain Lists}
\label{sec:org738e98b}
\begin{enumerate}
\item 3 种列表
\label{sec:org50fc74e}
\begin{itemize}
\item 无序列表
以 ``-'', ``+'', ``*'' 开头
\item 有序列表
以 ``1.'', ``1)'' 开头
\item 描述列表
以 ``::'' 将术语与描述分开
example: aaa :: bbb
\end{itemize}
\item 列表操作
\label{sec:org0215f6c}
\begin{itemize}
\item 列表内容可以用 TAB 来显示和折叠
\item M-RET
插入新项目
\item M-S-RET
插入有复选框的新项目
\item M-S-Up / M-S-Down
向上 / 向下移动项目包括子项目
\item M-Left / M-Right
减少和增加缩进
\item M-S-Left / M-S-Right
减少和增加缩进包括子项目
\item C-c C-c
切换复选框的状态
\item C-c -
循环切换列表使用的符号
\end{itemize}
\end{enumerate}
\subsection{表格 table}
\label{sec:orgb37c286}
\subsubsection{Intro}
\label{sec:org6eb8c90}
在表格中使用 TAB or RET or C-c C-c
可以重新对齐表格

完成表格的第一行后 C-c RET 即可创建
表格
\subsubsection{创建表格和移动单元格}
\label{sec:org48441eb}
\begin{itemize}
\item C-c |
可以创建表格也可以将符合标准的内容转换
成表格
\item S-Up / Down / Left / Right
移动单元格(通过交换相邻两个单元格)
\end{itemize}
\subsubsection{列操作}
\label{sec:org9939f59}
\begin{itemize}
\item M-Left,M-Right
向左,向右移动当前列
\item M-S-Left
Kill the current column
\item M-S-Right
在左侧插入一个新列
\end{itemize}
\subsubsection{行操作}
\label{sec:orgd213c3b}
\begin{itemize}
\item M-Up, M-Down
向上,向下移动当前行
\item M-S-Up
Kill the current row or horizontal line
\item M-S-Down
在上方插入新行
\item C-c -
在下方插入水平线
\item C-c RET
在下方插入水平线,并将光标移动到水平线下方
\end{itemize}
\subsubsection{表格排序}
\label{sec:orgf341ef2}
\begin{itemize}
\item C-c \^{}
\end{itemize}
\subsection{链接 Hyperlinks}
\label{sec:org6c66eca}
\subsubsection{两种基本的连接形式}
\label{sec:org30c7776}
\begin{verbatim}
[[LINK][DESCRIPTION]]
or
[[LINK]]
\end{verbatim}

\subsubsection{内部链接}
\label{sec:orgc3795c7}
\begin{enumerate}
\item 第一种
\begin{verbatim}
    [[#my-custom-id]]
    这个链接 --> "CUSTOM_ID"属性为 my-custom-id 的标题
\end{verbatim}
\item 第二种
\begin{verbatim}
   [[target]] or [[target][Click to target]]
   --> <<target>>
\end{verbatim}
\end{enumerate}

\subsubsection{外部链接}
\label{sec:orgf51d278}
\begin{verbatim}
‘http://www.astro.uva.nl/=dominik’	    on the web
‘file:/home/dominik/images/jupiter.jpg’	file, absolute path
‘/home/dominik/images/jupiter.jpg’	    same as above
‘file:papers/last.pdf’	                file, relative path
‘./papers/last.pdf’	                    same as above
‘file:projects.org’	                    another Org file
‘docview:papers/last.pdf::NNN’	        open in DocView mode at page NNN
‘id:B7423F4D-2E8A-471B-8810-C40F074717E9’	link to heading by ID
‘news:comp.emacs’	                        Usenet link
‘mailto:adent@galaxy.net’	                mail link
‘mhe:folder#id’	                        MH-E message link
‘rmail:folder#id’	                        Rmail message link
‘gnus:group#id’	                        Gnus article link
‘bbdb:R.*Stallman’	                    BBDB link (with regexp)
‘irc:/irc.com/#emacs/bob’	                IRC link
‘info:org#Hyperlinks’	                    Info node link

‘file:~/code/main.c::255’	                Find line 255
‘file:~/xx.org::My Target’	            Find ‘<<My Target>>’
‘[[file:~/xx.org::#my-custom-id]]’	        Find entry with a custom ID
\end{verbatim}
\subsubsection{处理链接}
\label{sec:org7547868}
\begin{itemize}
\item C-c C-l
编辑链接
插入链接
\item C-c C-o
打开链接
\item C-c \&
跳回记录点
记录点使用 C-c \% 标记
\end{itemize}

\subsection{待办 TODO Items}
\label{sec:org4ca4e5f}
\subsubsection{基本 TODO 功能}
\label{sec:org0e4ee7e}
以 TODO 开头的标题是一个 TODO

\begin{itemize}
\item C-c C-t
选择 TODO 状态
\item S-Right,S-Left
切换 TODO 状态
\item C-c | t
查看稀疏树中的 TODO
\item M-x org-agenda t
全局 TODO List
\item S-M-RET
插入新 TODO
\end{itemize}

\subsubsection{多状态工作流}
\label{sec:org2aea1d2}
doom-emacs 默认

\subsubsection{进度日志记录}
\label{sec:org38c6e04}
当任务变化时,进行的操作

\begin{enumerate}
\item 任务完成 DONE
\label{sec:org5f74535}
\begin{verbatim}
(setq org-log-done 'time) ; 任务为DONE时,添加一个"CLOSED:[timestamp]"
(setq org-log-done 'note) ; 任务为DONE时,添加一个"CLOSED:[timestamp]" 和 note
\end{verbatim}

\item 跟踪 TODO 状态变化
\label{sec:orgc37f033}
\begin{verbatim}
#+TODO: TODO(t) WAIT(W@/!) | DONE(d!) CANCELED(c@)
\end{verbatim}

\begin{center}
\begin{tabular}{ll}
标记 & 作用\\[0pt]
\hline
``!'' & 时间戳\\[0pt]
``@'' & note\\[0pt]
\end{tabular}
\end{center}
\end{enumerate}

\subsubsection{优先级}
\label{sec:orgc2356f8}
org 有 3 个优先级 A B C
B 是默认值

\begin{itemize}
\item C-c ,
设置优先级
\item S-Up, S-Down
增加,减少优先级
\end{itemize}

\subsubsection{将任务分解成子任务}
\label{sec:orga18335b}
在标题后面添加 ``[/]'' ``[\%]''

\subsubsection{复选框}
\label{sec:org4e6ac79}
\begin{itemize}
\item M-S-RET
insert checkbox
\item C-c C-c
toggle checkbox status

\item[{$\boxminus$}] item1[1/3]
\begin{itemize}
\item[{$\boxminus$}] item1-1 [50\%]
\begin{itemize}
\item[{$\square$}] item1-1-1
\item[{$\boxtimes$}] item1-1-2
\end{itemize}
\item[{$\boxtimes$}] item1-2
\item[{$\square$}] item1-3
\end{itemize}
\end{itemize}

\subsection{标签 Tags}
\label{sec:org1a064eb}
\subsubsection{标签的形式}
\label{sec:orga81c738}
每个标题都可以有一个标签列表

标签包含字母,数字,下划线,@

\begin{verbatim}
:work:

:work:urgent:
\end{verbatim}
\subsubsection{标签的继承}
\label{sec:org7c1603f}
\begin{enumerate}
\item 子标题会继承父标题的 Tags
\item 设置文件中所有标题都继承的 Tags
\begin{verbatim}
   #+FILETAGS: :Peter:Boss:Secret:
\end{verbatim}
\end{enumerate}
\subsubsection{设置标签\hfill{}\textsc{Keybndings}}
\label{sec:orgdb9b893}
\begin{itemize}
\item C-c C-q
插入标签
\item C-c C-c
同上,但需要焦点在标题上
\end{itemize}

设置默认的标签列表:
\begin{verbatim}
#+TAGS: @work @home
#+TAGS: laptop car pc sailboat
#+TAGS: @work(w) @home(h) laptop(l) pc(p)
\end{verbatim}
\subsubsection{标签组}
\label{sec:orgcee02b0}
\begin{verbatim}
#+TAGS: [ GTD : Control Persp ]
#+TAGS: { Context : @Home @Work }
\end{verbatim}
\subsubsection{搜索标签}
\label{sec:org2e00f94}
\begin{itemize}
\item C-c | m or C-c |
创建一个稀疏树,其中所有标题都匹配标签
搜索。使用前缀参数,忽略不是 TODO 的标题
\item M-x org-agenda m
从所有议程文件(agenda files)中创建标签匹配的全局列表
\item M-x org-agenda M
从所有议程文件中创建标签匹配的全局列表,但仅检查 TODO 项目并强制检查子项目
\end{itemize}

\subsection{属性 Properties}
\label{sec:org170b268}
\subsubsection{Intro}
\label{sec:org5b4cd1a}
:属性 1:    66666666
\begin{verbatim}
    :PROPERTIES:
    :Title:     Goldberg Variations
    :Composer:  J.S. Bach
    :Publisher: Deutsche Grammophon
    :NDisks:    1
    :END:
\end{verbatim}

可以定义 Xyz\textsubscript{ALL} 来定义 Xyz 的允许值

设置可以被文件中的任何标题继承的属性:
\begin{verbatim}
#+PROPERTY: NDisks_ALL 1 2 3 4
\end{verbatim}

\subsubsection{Keybinding}
\label{sec:orgc16d59b}
:属性 1:    66666666
\begin{itemize}
\item C-c C-x p or SPC m o
设置属性
\item C-c C-c d
从当前标题中删除属性
\end{itemize}

\subsection{日期和时间 Dates and Times}
\label{sec:orgbe3f332}

\subsubsection{时间戳 Timestamps}
\label{sec:org27dce46}
时间戳
\begin{verbatim}
<2003-09-16 Tue> or <2003-09-16 Tue 09:39> or <2003-09-16 Tue 12:00-12:30>
\end{verbatim}

\begin{enumerate}
\item 普通时间戳
\label{sec:org5e6e9e6}
一个简单的时间戳只是为一个项目分配一个日期/时间。

\item 有重复间隔的时间戳
\label{sec:orgc7ad231}
\begin{verbatim}
<2007-05-16 Wed 12:30 +1w>
\end{verbatim}
在特定间隔后仍然应用

间隔:
\begin{itemize}
\item d
days
\item w
weeks
\item m
months
\item y
years
\end{itemize}

\item 日记式的格式 Diary-style expression entries
\label{sec:org03d6c36}

对于更复杂的日期范围,Org mode 支持在 emacs 日历中实现特殊表达式日记条目。例如,可选时间:
\begin{verbatim}
    *  22:00-23:00 The nerd meeting on every 2nd Thursday of the month
        <%%(diary-float t 4 2)>
\end{verbatim}

\item 时间范围
\label{sec:org0dd7fb1}
由 “--” 连接的两个时间戳表示一个范围

\item 非活动时间戳
\label{sec:org9b25b82}
使用方括号包裹时间戳,这些时间戳不会触发
\end{enumerate}

\subsubsection{创建时间戳 Creating Timestamps}
\label{sec:orgf96bf22}
\begin{center}
\begin{tabular}{ll}
Keybinding & Action\\[0pt]
\hline
C-c . & 插入时间戳,修改时间戳,使用两次添加时间范围\\[0pt]
C-c ! & 和上一条一样,但是插入的是不活跃的时间戳\\[0pt]
S-Left,S-Right & 将日期更改一天\\[0pt]
S-Up,S-Down & 在尖括号处,更改时间戳类型;在时间戳中,更改年、月、日等\\[0pt]
\end{tabular}
\end{center}
\subsubsection{截止日期和日程安排 Deadlines and Scheduling}
\label{sec:org2e0519e}

\begin{center}
\begin{tabular}{ll}
Keybinding & Action\\[0pt]
\hline
C-c C-d & 插入 Deadline 和时间戳(预计的项目截止时间)\\[0pt]
C-c C-s & 插入 Scheduled 和时间戳(预计的项目开始时间)\\[0pt]
\end{tabular}
\end{center}
\subsubsection{Clocking Work Time}
\label{sec:org30b6f87}

工作时间计时

记录的是从开始时间到结束时间
\begin{itemize}
\item clock-in
开始计时(签到)
\item clock-out
停止计时(下班)
\end{itemize}

\begin{center}
\begin{tabular}{lll}
Keybinding & Action & Doom-Keybinding\\[0pt]
\hline
C-c C-x C-i & 开始计时(签到) & SPC m c i\\[0pt]
C-c C-x C-o & 停止计时(下班) & SPC m c o\\[0pt]
C-c C-x C-e & 更新工作量估计 & SPC m c e\\[0pt]
C-c C-x C-q & 取消时钟(开始计时后) & SPC n C\\[0pt]
C-c C-x C-j & 跳转到打卡任务的标题 & \\[0pt]
\end{tabular}
\end{center}

\subsection{Capture,Refile,Archive 捕获,重新存档,存档}
\label{sec:org3b59f85}

\subsubsection{Capture}
\label{sec:org699c82a}
快速笔记,可以定义模板

\begin{enumerate}
\item Setting up capture
\label{sec:org60827d1}
\begin{verbatim}
(setq org-default-notes-file (concat org-directory "path/to/filename.org"))
\end{verbatim}

\item Using capture
\label{sec:org64b39dd}
\begin{center}
\begin{tabular}{lll}
Keybinding & Action & Doom-Keybinding\\[0pt]
\hline
M-x org-capture & start capture & SPC X\\[0pt]
C-c C-c & return to before & \\[0pt]
C-c C-w & 通过将笔记重新归档到不同的地方来完成捕获 & \\[0pt]
C-c C-k & abort the capture and return to the previous state & \\[0pt]
\end{tabular}
\end{center}

\item Capture templates
\label{sec:orgbc28ec6}
\begin{verbatim}
(setq org-capture-templates
        '(("t" "Todo" entry (file+headline "path/to/todo.org" "Tasks")
            "* TODO %?\n %i\n %a")
          ("j" "Journal" entry (file+datetree "path/to/journal.org")
           "* %?\nEntered on %U\n  %i\n  %a")))
\end{verbatim}

动态插入的内容:
\begin{itemize}
\item \%a
注释,通常是用 org-store-link 创建的连接
\item \%i
初始内容,调用捕获时的区域
\item \%t,\%T
时间戳
\item \%u,\%U
不活跃的时间戳
\item \%?
加载模板完成后,光标的位置
\end{itemize}
\end{enumerate}


\subsubsection{Refile and Copy}
\label{sec:orgfe8eb23}

将 Capture 的数据重新归档或复制到不同的列表中,例如将 Capture 的数据复制到项目中。

\begin{center}
\begin{tabular}{ll}
Keybinding & Action\\[0pt]
\hline
C-c C-w & Refile the entry or region at piont\\[0pt]
C-u C-c C-w & Use the refile interface to jump to a heading\\[0pt]
C-u C-u C-c C-w & Jump to the location where org-refile last moved a tree to\\[0pt]
C-c M-w & Copying works like refiling,except that the original note is note deleted\\[0pt]
\end{tabular}
\end{center}

\subsubsection{Archiving}
\label{sec:org6ea6cbf}
When a project represented by a (sub)tree is finished, you may want to move the tree out of the way and to stop it from contributing to the agenda.

当一个(子)树代表的项目完成时,你可能想把这个树移出并停止它参与议程。

\begin{center}
\begin{tabular}{ll}
Keybinding & Action\\[0pt]
\hline
C-c C-x C-a & 使用变量 org-archive-default-command 中指定的命令归档当前条目\\[0pt]
C-c C-x C-s or short C-c \$ & 将子树从光标位置开始归档到 org-archive-location 给定的位置\\[0pt]
\end{tabular}
\end{center}

默认存档位置与当前文件位于同一目录,名称是``\textsubscript{archive}''与文件名连接而成。

\subsection{Agenda Views 议程视图}
\label{sec:orgf5fd282}

\subsubsection{Agenda Files}
\label{sec:orgbc821fe}
\begin{center}
\begin{tabular}{ll}
Keybinding & Action\\[0pt]
\hline
C-c [ & 将当前文件添加到议程文件列表中\\[0pt]
C-c ] & 将当前文件移除到议程文件列表中\\[0pt]
C-c ' or C-c , & 循环浏览议程文件列表,依次展示文件\\[0pt]
\end{tabular}
\end{center}
\subsubsection{The Agenda Dispatcher}
\label{sec:org1e77462}
\begin{itemize}
\item DOOM-Keybinding
\begin{verbatim}
open the agenda dispatcher
SPC o A
\end{verbatim}
\end{itemize}
\subsubsection{The Weekly/Daily Agenda}
\label{sec:org3300975}
时间戳可以触发
\subsubsection{The Global TODO List}
\label{sec:org493f306}
SPC o A t or SPC o A T

\subsubsection{Commands in Agenda Buffer}
\label{sec:org99157e1}
\begin{itemize}
\item t
\begin{verbatim}
下一行
\end{verbatim}
\item p
\begin{verbatim}
上一行
\end{verbatim}
\item SPC
\begin{verbatim}
在另一个窗口中显示项目的原始位置
\end{verbatim}
\item TAB
\begin{verbatim}
转到项目在另一个窗口中的原始位置
\end{verbatim}
\item RET
\begin{verbatim}
转到项目的原始位置并删除其他窗口
\end{verbatim}
\item o
\begin{verbatim}
删除其他窗口
\end{verbatim}
\item d
\begin{verbatim}
切换到日视图
\end{verbatim}
\item w
\begin{verbatim}
切换到周视图
\end{verbatim}
\item f
\begin{verbatim}
显示进度
\end{verbatim}
\item b
\begin{verbatim}
显示更早的日期
\end{verbatim}
\item .
\begin{verbatim}
今天
\end{verbatim}
\item j
\begin{verbatim}
日期提示,并切换到该日期
\end{verbatim}
\item l
\begin{verbatim}
切换日志模式
\end{verbatim}
\item r
\begin{verbatim}
重新创建缓冲区
\end{verbatim}
\item s
\begin{verbatim}
保存
\end{verbatim}
\end{itemize}

\subsection{Markup for Rich Contents}
\label{sec:org6f125f6}

\subsubsection{Paragraphs}
\label{sec:org7304d2f}
Paragraphs are separated by at least one empty line.

Use ``$\backslash$\'' at the end of a line to enforce a line break within a paragraph.

Format poetry:
\begin{verse}
第一行\\[0pt]
第二行\\[0pt]
\hspace*{4em}--someone\\[0pt]
\end{verse}

Center:
\begin{center}
Hello World!
\end{center}

\subsubsection{Emphasis and Monospace 强调和等宽}
\label{sec:orge41881d}
\begin{verbatim}
*bold*
/italic/
_unerlined_
=verbatim=
~code~
+strike-through+
\end{verbatim}

\textbf{bold}
\emph{italic}
\uline{unerlined}
\texttt{verbatim}
\texttt{code}
\sout{strike-through}

\subsubsection{Embedded \LaTeX{}}
\label{sec:org144e2f0}
The radius of the sun is R\textsubscript{sun} = 6.96 x 10\textsuperscript{8} m.  On the other hand,
the radius of Alpha Centauri is R\textsubscript{Alpha Centauri} = 1.28 x R\textsubscript{sun}.

\begin{equation}                        \% arbitrary environments,
x=\sqrt{b}                              \% even tables, figures
\end{equation}                          \% etc

If \(a^2=b\) and \(b=2\), then the solution must be
either $$ a=+\sqrt{2} $$ or \[ a=-\sqrt{2} \].
\subsubsection{Literal example}
\label{sec:orge1d3fda}
示例,使用等宽字体,不受标记影响
\begin{verbatim}
this is a exmaple
\end{verbatim}

Here is an example
\begin{verbatim}
Some example from a note.org
another line
\end{verbatim}
\subsubsection{Images}
\label{sec:orgfbb08cf}
图像链接
\begin{verbatim}
[[./link/to/image.png]]
\end{verbatim}


定义图片的标题和引用标签
\begin{figure}[htbp]
\centering
\includegraphics[width=.9\linewidth]{./image.png}
\caption{\label{test}This is the caption for the next image link (or table)}
\end{figure}

\begin{itemize}
\item Doom-Keybinding: z i
toggle the display of image
\end{itemize}
\subsubsection{Creating Footnotes}
\label{sec:orga18c706}
脚注的定义
\begin{verbatim}
this is a footnotes[fn:1]
...
[fn:1] I am footnote
\end{verbatim}


\begin{center}
\begin{tabular}{ll}
Keybinding & Action\\[0pt]
\hline
C-c C-x f or SPC m f & 脚注的相关操作\\[0pt]
C-c C-c & 在定义和引用之间跳转\\[0pt]
\end{tabular}
\end{center}

\subsection{Exporting}
\label{sec:orgcbb1778}

\subsubsection{The Export Dispatcher}
\label{sec:org2a1127c}
C-c C-e \\[0pt]
SPC m e
\subsubsection{Export Settings}
\label{sec:org97f75b3}
\begin{verbatim}
#+title: this is title
#+author: default taken from user-full-name
#+date: a date, fixed, or an Org timestamp
#+email: default from user-mail-address
#+LANGUAGE: language code
\end{verbatim}

\subsubsection{Table of Contents 目录}
\label{sec:org72e4022}
\begin{verbatim}
#+options: toc:2    (only include two levels in TOC)
#+options: toc:nil  (no default TOC at all)
\end{verbatim}
\subsubsection{Inculude Files}
\label{sec:org1a46a46}
\begin{verbatim}
#+include: "path/to/file" src emacs-lisp
\end{verbatim}
第一个参数是文件路径,第二个参数是指定块的类型,第三个参数指定格式化内容的源代码语言。

使用 C-c ' 可以访问包含的文件

\subsubsection{Comment Lines}
\label{sec:orgb66fe88}
\begin{verbatim}
# this is comment

#+begin_comment
this is comment
#+end_comment
\end{verbatim}

\begin{itemize}
\item C-c ;
将标题前加 COMMENT,会注释掉子树
\end{itemize}
\end{document}
